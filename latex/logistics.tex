\documentclass[UTF8]{ctexart}
\usepackage{amsmath}
\usepackage{amssymb}
\usepackage{graphics}
\usepackage{geometry}
\geometry{left=2.5cm,right=2.5cm,top=2.5cm,bottom=2.5cm}
\begin{document}

%\begin{equation}
%	\forall i \in N ~ \sum_{k=1}^{K}\sum_{j=1}^{2} X_{ijk}=1
%\end{equation}

\section{问题重述}
\subsection{提出问题}
	近年来我国汽车市场每年平均以25%的速度快速增长,其发展速度快于国民经济的增长。汽车消费需求的增长,促进了乘用车的整车物流量迅速上升。乘用车生产厂家根据全国客户的购车订单,向物流公司下达运输乘用车到全国各地的任务,物流公司则根据下达的任务制定运输计划并配送这批乘用车。为此,物流公司首先要从他们当时可以调用的“轿运车”中选择出若干辆轿运车,进而给出其中每一辆轿运车上乘用车的装载方案和目的地,以保证运输任务的完成。
但因轿运车、乘用车规格的多样性和运输路线的多样性,当前很多物流公司在制定运输计划时主要依赖调度人员的经验。而在运输任务较复杂时,往往会发生运输效率低下,运输成本不理想等问题。物流公司希望能够在确保完成运输任务的前提下寻求降低运输成本的方法。
“轿运车”是通过公路来运输乘用车整车的专用运输车。本文涉及的双层轿运车分为三种子型:上下层各装载1列乘用车,故记为1-1型(图1);下、上层分别装载1、2列,记为1-2型(图2);上、下层各装载2列,记为2-2型(图3),每辆轿运车可以装载乘用车的最大数量在6到27辆之间。

\subsection{问题要求}
建立数学模型,以通用算法和程序解决整车物流成本优化的问题,能够比较准确得模拟整车物流的装载方案及行车路线,更好地降低整车物流运输成本。
问题一:物流公司要运输Ⅰ车型的乘用车100辆及Ⅱ车型的乘用车68辆。
问题二:物流公司要运输Ⅱ车型的乘用车72辆及Ⅲ车型的乘用车52辆。
问题三:物流公司要运输Ⅰ车型的乘用车156辆、Ⅱ车型的乘用车102辆及Ⅲ车型的乘用车39辆。
问题四:物流公司要运输166辆Ⅰ车型的乘用车(其中目的地是A、B、C、D的分别为42、50、33、41辆)和78辆Ⅱ车型的乘用车(其中目的地是A、C的,分别为31、47辆),具体路线见图4,各段长度:OD=160,DC=76,DA=200,DB=120,BE=104,AE=60。
问题五:附件的表1给出了物流公司需要运输的乘用车类型(含序号)、尺寸大小、数量和目的地,附件的表2给出可以调用的轿运车类型(含序号)、数量和装载区域大小(表里数据是下层装载区域的长和宽, 1-1型及2-2型轿运车上、下层装载区域相同;1-2型轿运车上、下层装载区域长度相同,但上层比下层宽0.8米。此外2-2型轿运车因为层高较低,上、下层均不能装载高度超过1.7米的乘用车。

\section{模型假设}
\begin{enumerate}
	\item 	题目中所列数据均真实可靠;
	\item	乘用车与轿运车两侧的安全距离在说明装载区域大小时已经考虑;
	\item	乘用车不可在轿运车上倾斜摆放
	\item	轿运车运输路线为有向图
	\item	运输花费主要由轿运车数量决定,次要由轿运车类型决定

\end{enumerate}


\end{document}